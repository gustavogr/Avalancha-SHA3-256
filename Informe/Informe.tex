%%%%%%%%%%%%%%%%%%%%%%%%%%%%%%%%%%%%%%%%%
% University/School Laboratory Report
% LaTeX Template
% Version 3.1 (25/3/14)
%
% This template has been downloaded from:
% http://www.LaTeXTemplates.com
%
% Original author:
% Linux and Unix Users Group at Virginia Tech Wiki 
% (https://vtluug.org/wiki/Example_LaTeX_chem_lab_report)
%
% License:
% CC BY-NC-SA 3.0 (http://creativecommons.org/licenses/by-nc-sa/3.0/)
%
%%%%%%%%%%%%%%%%%%%%%%%%%%%%%%%%%%%%%%%%%

%----------------------------------------------------------------------------------------
%   PACKAGES AND DOCUMENT CONFIGURATIONS
%----------------------------------------------------------------------------------------

\documentclass[a4paper]{article}

\usepackage[utf8]{inputenc}
\usepackage[spanish]{babel}
\usepackage{graphicx} % Required for the inclusion of images
\usepackage{natbib} % Required to change bibliography style to APA
\usepackage{hyperref}

\usepackage{floatrow}
\newfloatcommand{capbtabbox}{table}[][\FBwidth]

\usepackage{xcolor}
\usepackage[normalem]{ulem}
\useunder{\uline}{\ulined}{}%
\DeclareUrlCommand{\blurl}{\def\UrlFont{\ttfamily\color{blue}\ulined}}

\setlength\parindent{0pt} % Removes all indentation from paragraphs

\renewcommand{\labelenumi}{\alph{enumi}.} % Make numbering in the enumerate environment by letter rather than number (e.g. section 6)

%\usepackage{times} % Uncomment to use the Times New Roman font

%----------------------------------------------------------------------------------------
%   DOCUMENT INFORMATION
%----------------------------------------------------------------------------------------

\title{\includegraphics[width=0.35\textwidth]{UPM} \\ 
Estudio del efecto Avalancha \\ mostrado por la función Hash \\ SHA-3 (256)} % Title

\author{Gustavo Gutiérrez \\ 50588464X} % Author name

\date{\today} % Date for the report

\begin{document}


\maketitle % Insert the title, author and date

\begin{center}
\begin{tabular}{l r}
Fecha del experimento: & 30 de octubre de 2016 \\
Profesor: & Jorge Dávila % Instructor/supervisor
\end{tabular}
\end{center}

\clearpage

\tableofcontents{}

\clearpage

%----------------------------------------------------------------------------------------
%   SECTION 1
%----------------------------------------------------------------------------------------

\section{Introducción}

Se conoce como efecto Avalancha a la propiedad deseada de una función criptográfica donde pequeños cambios sobre los argumentos
de entrada provocan grandes cambios en el resultado de la función\footnote{\blurl{https://en.wikipedia.org/wiki/Avalanche_effect}}.
El algoritmo a estudiar en este experimento, la función hash criptográfica SHA-3 (256), cuenta con un sólo argumento de entrada 
que es la cadena a la cual se le calculará el hash.

\section{Implementación}

El experimento fue realizado utilizando el lenguaje de propósito general Python\footnote{\blurl{https://docs.python.org/3/}}. 
Se utilizó la libreria \emph{Pysha3}\footnote{\blurl{https://pypi.python.org/pypi/pysha3}} para la implementación del algoritmo
de cifrado. Asimismo se utilizó la librería \emph{Bitstring}\footnote{\blurl{https://pypi.python.org/pypi/bitstring/3.1.3}} para 
facilitar el manejo de de las cadenas de bits. Para garantizar el correcto funcionamiento de la primitiva criptográfica se 
probó la función con vectores de prueba\footnote{\blurl{http://www.di-mgt.com.au/sha_testvectors.html}}. 

\subsection{Estructura de una corrida}
Lo primero a hacer en una corrida del experimento es generar las cadenas de bits a cifrar. Para ello se genera un numero 
entero aleatorio de 256 bits, se elige al azar uno de los 256 bits para cambiar y se genera el entero con el bit cambiado. 
Finalmente, antes de calcular el hash se deben convertir los enteros en su representación binaria para lo cual se utiliza 
la librería bitstring. Una vez calculados los hashes de las cadenas, se mide la distancia de Hamming 
entre los dos Hashes y se devuelve el valor para ser acumulado.
 
 \subsection{Significación del experimento y número de corridas a realizar}
 Para este experimento se decidió conseguir un resultado con significancia de 92.5\%. El número de corridas a realizar 
 se definió de manera experimental. Para ello se hace una cantidad arbitraria de corridas. Sobre este resultado se miden
 la varianza y la media y se comparan con los valores ideales (se explican en la sección de comportamiento esperado). Esta
 medición se realiza con tres lotes de corridas. Si en alguno de los lotes el error de las medidas es mayor a 7.5\% con 
 respecto a los valores ideales se aumenta el numero de pruebas a realizar. Para conseguir el nivel de significancia deseado
 se hicieron lotes de 1000 corridas.

\section{Resultados experimentales}

Para los resultados finales se tomaron los datos de los tres lotes y se interpretan como un solo experimento. Sobre estos 
datos se calculó la media, la mediana, la moda, la varianza y la desviación estándar de las distancias obtenidas. Para ello se
utilizaron las funciones estadísticas incluidas en \emph{Python}\footnote{\blurl{https://docs.python.org/3/library/statistics.html}}. 
La figura 1 muestra el histograma de frecuencias de aparición de cada distancia y el cuadro 1 muestra las medidas estadísticas 
obtenidas en el experimento.

\begin{figure}
\begin{floatrow}
\ffigbox{%
  \includegraphics[width=0.55\textwidth]{Histograma} %
}{%
  \caption{Histograma obtenido}%
}
\capbtabbox{%
  \begin{tabular}{l r}
    Media: & 127.934 \\
    Moda: & 128 \\
    Mediana: & 128 \\
    Desviacion: & 7.992 \\
    Varianza: & 63.882 
  \end{tabular}
}{%
  \caption{Resultados estadísticos}%
}
\end{floatrow}
\end{figure}

\section{Comparación con un Oráculo Aleatorio}

Si la función criptográfica se comportara como un Oráculo aleatorio\footnote{\blurl{https://en.wikipedia.org/wiki/Random_oracle}}
las dos cadenas de bits resultantes serían totalmente aleatorias. De ser este el caso, la posibilidad de que un determinado
bit sea distinto en las dos cadenas es de 0.5. Con esto en mente se puede ver como que la distancia de hamming entre dos cadenas es 
equivalente a realizar 256 elecciones de bits, con una probabilidad del 0.5 de que el bit sea distinto. Este comportamiento
es el descrito por una distribucion binomial\footnote{\blurl{https://en.wikipedia.org/wiki/Binomial_distribution}} con parámetros
n = 256 y p = 0.5. Por consiguiente la media de la distribución sería \textbf{128} (np) y la varianza \textbf{64} (np(p-1)).

\section{Conclusiones}

Como lo evidencian los resultados mostrados anteriormente la distancia de hamming entre los hashes de dos cadenas 
prácticamente identicas sigue una distribución casi idéntica al comportamiento esperado de un Oráculo Aleatorio. Por 
ende los resultados de este experimento nos llevan a afirmar que la funcion hash criptográfica \textbf{SHA-3 (256)} 
muestra un fuerte efecto avalancha. 

\end{document}